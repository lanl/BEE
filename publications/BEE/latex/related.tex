\section{Related Work}
\label{sec:RelatedWork}
Containers provide an opportunity to leverage cloud and industry software and practices within the HPC environment. However, various restrictions on current HPC system software and security policies limit the use of industry standard container technologies to execute applications on HPC platforms. Multiple efforts are underway within the DoE laboratories to provide solutions that ameliorate these issues (e.g., Shifter\cite{jacobsen2015contain}, Charliecloud\cite{kurtzer_2016_60736}, Singularity\cite{priedhorsky2016charliecloud}, and \texttt{BEE-VM}) and enable containers to execute securely on HPC platforms. 

Shifter is an execution environment that also aims to provide containerized environments for HPC systems. Since deploying the standard Docker daemon on HPC systems imposes security and compatibility issues, they build a Docker-like container environment, which provides portability, isolation, and reproducibility like Docker. Shifter runs customized Shifter images. Docker users need to first import their Docker images and convert them into Shifter images before running. Shifter containers can access the host file system via volume mapping; however, there is no explicit application data management. Also, sharing files between containers requires the file sharing abilities between host machines. Shifter can only be installed on customized Cray machines with root privileges, whereas \texttt{BEE} can run standard Docker images unmodified. Using \texttt{BEE} is easier for developers to ensure consistent environments across their local development and test machines along with production systems. Using standard Docker brings more convenience to develop and distribute Docker images in Docker communities. \texttt{BEE} has explicit application data management that can facilitate easy transfer across host machines for live migration or work flow integration. Also, \texttt{BEE-VM} has several modes for data file sharing between processes that can be configured by the user depending on whether file sharing is enabled on the hosts. If not, \texttt{BEE-VM} can build its own file sharing mechanism, which brings more flexibility. Finally, \texttt{BEE-VM} can be deployed on any HPC system and even cloud systems without root privileges. 

Singularity is another containerized execution environment for HPC systems. Similar to Shifter, it also build a Docker-like container execution environment to run customized Singularity images. Standard Docker images are supported but they also need to be converted to Singularity format before running. It is also required to have root privileges in order to install Singularity on HPC systems. Unlike Shifter, Singularity can be deployed on any HPC system. It does not seek to manage application data. Data sharing between containers also depends on the host filesystems. With multiple configuration solutions, \texttt{BEE} has more flexibility for deployment on HPC systems. Besides providing a containerized execution environment, \texttt{BEE} brings better usability by combining data management, workflow integration, live migration, and cloud computing together to provide a more convenient tool for HPC users and developers.

Charliecloud is a container solution that brings Docker-composed environments into HPC systems. It brings many of the benefits of standard Docker container; the main benefit  being that users or developers can have consistent building and execution environment from their local development and test machines to large scale production cluster machines. However, it is challenging to install Charliecloud on current HPC systems. Installing Charliecloud environment requires that the target HPC system has a Linux kernel version of at least 3.18.  This limitation constrains the current production HPC system ability to support Charliecloud. Moreover, it is only designed as the execution environment, many other aspects of HPC workflow have not been integrated.

While this paper details a specific \texttt{BEE} backend implementation, the \texttt{BEE-VM} and
\texttt{BEE-AWS}, there is nothing that precludes the addition of \texttt{BEE-Shifter}, \texttt{BEE-Singularity}, or \texttt{BEE-Charliecloud} backends for specific platforms.

Amazon EC2 Container Service (ECS) \cite{awscontainer} is a Docker container service provided by Amazon on AWS. Since ECS deploys on top of EC2, and EC2 has a layer on VMs, ECS actually deploys the Docker container layer on top of the VM layer, providing a similar host-VM-Docker structure as \texttt{BEE-VM}. Regardless of the underlying hardware configuration, ECS provides a consistent building and execution environment by using standard \texttt{BEE}. Docker users can easily run their application on ECS without modification. This allows us to deploy the similar structure on both cloud and HPC environment in \texttt{BEE}. 
  

